\documentclass[12pt,letterpaper]{book}

\usepackage{xcolor}

\usepackage{tocbasic}
%\usetocstyle{standard}


\renewcommand{\today}{February 15-17, 2023}

\usepackage[a4paper,margin=3cm,innermargin=3cm]{geometry}

\usepackage[
 type={CC},
 modifier={by-nc-nd},
 version={4.0},
]{doclicense} 

\usepackage{needspace}
\usepackage{marginnote}
\usepackage{ifxetex}
\renewcommand*{\marginfont}{\sffamily\footnotesize}

\usepackage{imakeidx}

\usepackage[hidelinks]{hyperref}
\hypersetup{
pdftitle={Book of Abstracts},
pdfsubject={WAMTA2024},
pdfauthor={Joseph Schuchart},
pdfkeywords={asynchronous many-task systems, applications, machine learning, artificial intelligence}
}
\makeindex[intoc]

\newenvironment{conf-abstract}[4][]{
 \needspace{10\baselineskip}
 \begin{center}
 { \renewcommand\textsuperscript[1]{}
 \phantomsection\addcontentsline{toc}{section}
 {\texorpdfstring{#2 (\emph{#3})}{#2 (#3)}}
 }
 {{\large\bfseries #2}\marginnote{#1}\par}
 \medskip
 {#3\par}
 \smallskip
 {\small #4\par}
 \end{center}
}{%
 \bigskip
 \hrule
 \bigskip
}

\usepackage{etoolbox}
\newcommand{\indexauthors}[1]{%
 \forcsvlist{\index}{#1}
}

\setcounter{tocdepth}{3}
\setcounter{secnumdepth}{-1}
\pagestyle{plain}


\title{Workshop on Asynchronous Many-Task Systems 2024}
\author{Venue: \\ Student Union Building\\University of Tennessee, Knoxville}

\ifxetex
\usepackage{fontspec}
\setmainfont{Raleway}
\fi

\usepackage{timetable/calendar} 

\usepackage{rotating}

\usepackage{datatool}

\DTLloaddb{data}{./data.csv}
\DTLloaddb{dataPoster}{./dataPoster.csv}


\newcommand*{\thevalue}{}
\newcommand*{\getCol}[2]{%
 \DTLgetvalueforkey{\thevalue}{Last}{data}{Talk}{#2}%
}

\newcommand*{\lastname}[1]{
\getCol{1}{#1}
\thevalue 
}


\newcommand*{\thevalues}{}
\newcommand*{\getCols}[2]{%
 \DTLgetvalueforkey{\thevalues}{Time}{data}{Talk}{#2}%
}



\newcommand*{\gettime}[1]{
\getCols{1}{#1}
\kern-1ex\thevalues
}




\newcommand*{\thetitlevalues}{}
\newcommand*{\getTitleCols}[2]{%
 \DTLgetvalueforkey{\thetitlevalues}{Title}{data}{Talk}{#2}%
}



\newcommand*{\gettitle}[1]{
\getTitleCols{1}{#1}
\kern-1ex\thetitlevalues
}



\begin{document}

\frontmatter

\maketitle

This workshop is sponsored by

\begin{itemize}
\item University of Tennessee, Knoxville
\item TODO
\end{itemize}

\newpage

\section*{Abstract}
As our compute capacity grows, science simulations are not only becoming bigger, but more complex. Simulations are carried out at multiple scales and using multiple kinds of physics at once. Boundaries are irregular, grids are irregular, computational domains can be dynamic and complex. In such scenarios, the ideal way to parallelize often cannot be statically determined. At the same time, hardware is becoming more heterogeneous and difficult to program. Increasingly, scientists are turning to asynchronous, dynamic parallelism in order to make the best use of increasingly challenging hardware. As a result, numerous frameworks, platforms, and specialized languages have sprung up to answer this need.

The objectives of this workshop are to bring together experts in asynchronous many-task frameworks, developers of science codes, performance experts, and hardware vendors to discuss the state-of-the-art techniques needed to program, analyze, benchmark, and profile these codes to achieve maximum performance possible from modern machines. This workshop will promote a dialogue between these communities, and help identify challenges and opportunities for advancement in all the disciplines they represent.

\section*{Organizing committee}
\begin{itemize}
\item Patrick Diehl, Louisiana State University (USA)
\item Pedro Valero-Lara, Oak Ridge National Laboratory (USA)
\item George Bosilca, NVIDIA (USA)
\item Joseph Schuchart, University of Tennessee, Knoxville (USA)
\end{itemize}

\section*{Scientific committee}
\begin{itemize}
\item Alex Aiken, Stanford (USA)
\item Erwin Laure, Max Planck Computing \& Data Facility (Germany)
\item Christoph Junghans, Los Alamos National Laboratory (USA)
\item Bryce Adelstein Lelbach, NVIDIA (USA)
\item  Laxmikant V. Kale, University of Illinois at Urbana-Champaign (USA)
\item Brad Chamberlain, HPE and University of Washington (USA)
\end{itemize}

\section*{Technical program chair}

\begin{itemize}
\item Patrick Diehl, Louisiana State University (USA)
\item Pedro Valero-Lara, Oak Ridge National Laboratory (USA)
\item George Bosilca, NVIDIA (USA)
\item Joseph Schuchart, University of Tennessee, Knoxville (USA)
\end{itemize}

\section*{Technical program }

\begin{itemize}
\item Brad Richardson, Sourcery Institute (USA)
\item Kevin Huck, University of Oregon (USA)
\item Dirk Pflüger, University of Stuttgart (Germany)
\item Metin H. Aktulga, Michigan State University (USA)
\item Huda Ibeid, Intel
\item Dirk Pleiter, KTH Royal Institute of Technology (Sweden)
\item Didem Unat, Koç University (Turkey)
\item Keita Teranishi, Sandia National Laboratories (USA)
\item Gregor Daiß, University of Stuttgart (Germany)
\item Najoude Nader, Louisiana State University (USA)
\item Weile Wei, Lawrence Berkeley National Laboratory (USA)
\item Jeff Hammond, NVIDIA (Finland)
\item Hartmut Kaiser, Louisiana State University (USA)
\item J. Ram Ramanujam, Louisiana State University (USA)
\item Steven R. Brandt, Louisiana State University (USA)
\item Narasinga Rao Miniskar, Oak Ridge National Laboratory (USA)
\item Markus Rampp, Max Planck Computing and Data Facility (Germany)
\item Sumathi Lakshmiranganatha, Los Alamos National Laboratory (USA)
\item Nikunj Gupta, Amazon (USA)
\item Jonas Posner, University of Kassel (Germany)
\end{itemize}


\section*{Logistics}
\begin{itemize}
\item Parsa Aminim, Halpern-Wight Inc. (USA)
\item Joseph Schuchart and Thomas Herault, University of Tennessee, Knoxville (USA)
\end{itemize}



TODO

\chapter{Welcome Address}
It is my distinct pleasure to welcome you all to the Workshop on Asynchronous Many-task Systems and Applications (February 15-17, 2023) held at the Center for Computation and Technology (CCT) on the Louisiana State University campus in Baton Rouge, Louisiana.\\

\noindent I am pleased to be able to support and co-sponsor this workshop along with a number of others: Tactical Computation Labs, the National Science Foundation, and Hewlett Packard Enterprise. We at CCT greatly appreciate the support of our co-sponsors. I would like to thank the great work of the organizers of this workshop, Dr.\ Patrick Diehl, Dr.\ Hartmut Kaiser, Dr.\ Gerald Baumgartner, and Dr.\ Steven R. Brandt. In addition, special thanks to Dr.\ Robert Twilley, Interim Vice-President of Research at LSU for the continuing strong support of CCT.\\

\noindent The objectives of this workshop are to bring together experts in asynchronous many-task frameworks, developers of science codes, performance experts, and hardware vendors to discuss the state-of-the-art techniques needed to program, analyze, benchmark, and profile these codes to achieve maximum performance possible on modern machines. This workshop will promote a dialogue between these communities and help identify challenges and opportunities for advancement in all the disciplines they represent.\\

\noindent I would like to thank the three excellent keynote speakers, Prof.\ Michelle Strout (University of Arizona and HPE), Dr.\ Damian Rouson Lawrence Berkeley National Laboratory), and Prof.\ George Bosilca (University of Tennessee, Knoxville), who are world-class experts in the field, for sharing their thoughts. In addition, I thank all the participants.\\

\noindent I sincerely hope you enjoy and benefit from this unique workshop and the discussion sessions during the coming two and a half days.\\

\noindent With best wishes,

J. “Ram” Ramanujam



%\begin{sidewaysfigure}
%\begin{calendar}{\textwidth} % Calendar to be the entire width of the page

\setcounter{calendardate}{11} % Day on which the calendar starts - note that you have to account for blank days

%----------------------------------------------------------------------------------------
%	Monday
%----------------------------------------------------------------------------------------

\day{}{}

%----------------------------------------------------------------------------------------
%	SECOND DAY
%----------------------------------------------------------------------------------------
\day{}{}

%----------------------------------------------------------------------------------------
%	THIRD DAY
%----------------------------------------------------------------------------------------

\day{}{}

%----------------------------------------------------------------------------------------
%	FOURTH DAY
%----------------------------------------------------------------------------------------


\day{Day 1}{
\textbf{Registration} \\
\textbf{Coffee and Pastries} \\\vspace{0.1cm}
8:00 AM-8:45 AM \\\vspace{0.1cm}
Welcoming Remarks \\\vspace{0.1cm}8:45 AM-9:00 AM\\\daysep
\lastname{1}~ \\\vspace{0.1cm}\gettime{1} \\\vspace{0.5cm}\daysep
\textbf{Coffee} \\ 10:00 AM-10:30 AM \\\daysep
\lastname{2}~ \\\vspace{0.1cm} \gettime{2} \\\daysep
\lastname{3}~ \\\vspace{0.1cm} \gettime{3} \\\daysep
\lastname{4}~ \\\vspace{0.1cm} \gettime{4} \\\daysep
\textbf{Lunch} \\ 12:00 PM-1:30 PM \vspace{0.2cm} \\\daysep
\lastname{5}~ \\\vspace{0.1cm} \gettime{5} \\\daysep
\lastname{6}~ \\\vspace{0.1cm} \gettime{6} \\\daysep
\lastname{7}~ \\\vspace{0.1cm} \gettime{7} \\\daysep
\textbf{Coffee} \\ 3:00 PM-3:30 PM \\\daysep
\lastname{8}~ \\\vspace{0.1cm} \gettime{8} \\\daysep
\lastname{9}~ \\\vspace{0.1cm} \gettime{9} \\\daysep
\lastname{23}~ \\\vspace{0.1cm} \gettime{23} \\\daysep
\textit{Discussion ?} \\\vspace{0.1cm} 5:00 PM-5:30 PM \\\daysep
\textbf{Pizza \& Beer} \\ \textit{HiWire Brewing} \\\vspace{0.1cm} 7:00 PM--?? \\ (2020 Barber St, Knoxville, TN 37920) \\\daysep
}

%----------------------------------------------------------------------------------------
%	FIFTH DAY
%----------------------------------------------------------------------------------------

\day{Day 2}{
\textbf{Registration} \\
\textbf{Coffee and Pastries} \\\vspace{0.1cm}
8:00 AM-9:00 AM \\\vspace{0.1cm} \textcolor{white}{Welcome remarks} \\\vspace{0.1cm} \textcolor{white}{8:45 AM-9:00 AM} \\\daysep
\lastname{10}~ \\\vspace{0.1cm} \gettime{10} \\\vspace{0.5cm}\daysep
\textbf{Coffee} \\10:00 AM-10:30 AM \\\daysep
\lastname{11}~ \\\vspace{0.1cm} \gettime{11} \\\daysep
\lastname{12}~ \\\vspace{0.1cm} \gettime{12} \\\daysep
\lastname{13}~ \\\vspace{0.1cm} \gettime{13} \\\daysep
\textbf{Lunch} \\ 12:00 PM-1:30 PM \vspace{0.2cm}\\\daysep
\lastname{14}~ \\\vspace{0.1cm} \gettime{14} \\\daysep
\lastname{15}~ \\\vspace{0.1cm} \gettime{15} \\\daysep
\lastname{16}~ \\\vspace{0.1cm} \gettime{16} \\\daysep
\lastname{17}~ \\\vspace{0.1cm} \gettime{17} \\\daysep
\textbf{Coffee} \\3:30 PM-4:00 PM \\\daysep
\textit{Discussion?}  \\\vspace{0.1cm}4:00 PM-4:30 PM \\\daysep
\textbf{Poster session} \\\vspace{0.1cm} 4:30 PM--5:30 PM \\\daysep
\textbf{Banquet} \\
\textit{Calhoun's on the river}\\\vspace{0.1cm} \textbf{\textit{Industrial talk}} \\ Chris Taylor \\Tactical Computing Lab \\\vspace{0.1cm}  6:00 PM-9:00 PM \\ 400 Neyland Dr, Knoxville, TN 37902\\\daysep
%\textbf{Industrial talk} \\ Chris Taylor \\Tactical Computing Lab \\\vspace{0.1cm} The Club at LSU Union Square \\\vspace{0.1cm} 7:00 PM-10:00 PM\\\daysep
}

%----------------------------------------------------------------------------------------
%	SIXTH DAY
%----------------------------------------------------------------------------------------


\day{Day 3}{
\textbf{Registration} \\
\textbf{Coffee and Pastries} \\\vspace{0.1cm}
8:00 AM-9:00 AM \\\vspace{0.1cm} \textcolor{white}{Welcome remarks} \\\vspace{0.1cm} \textcolor{white}{8:45 AM-9:00 AM} \\\daysep
\lastname{18}~ \\\vspace{0.1cm} \gettime{18}  \\\vspace{0.5cm}\daysep
\textbf{Coffee}~ \\10:00 AM-10:30 AM \\\daysep
\lastname{19}~ \\\vspace{0.1cm} \gettime{19} \\\daysep
\lastname{20}~ \\\vspace{0.09cm} \gettime{20} \\\daysep
\lastname{21}~ \\\vspace{0.1cm} \gettime{21} \\\daysep
\lastname{22}~ \\\vspace{0.1cm} \gettime{22} \\\daysep
%\lastname{24}~ \\\vspace{0.1cm} \gettime{14} \\\daysep
%Final discussion (Fineberg) \\ 12:00 PM--12:30 PM \\\daysep
\textbf{Lunch} \\ 12:30 PM-2:00 PM \vspace{0.2cm} \\\daysep
}



%----------------------------------------------------------------------------------------
%	SEVENTH DAY
%----------------------------------------------------------------------------------------





% Note: more days can be added to give the calendar a third or fourth week

%----------------------------------------------------------------------------------------

\finishCalendar
\end{calendar}

%\end{sidewaysfigure}

\newpage

%\includegraphics[width=\linewidth]{timetable_edit.png}


\tableofcontents

\mainmatter

\chapter{Industrial talk}

TODO

\begin{conf-abstract}[27$^{th}$]
{An Azimuth to RISC-V HPC}
{Chris Taylor}
{Tactical Computing Lab}
\index{Taylor!Chris}
The RISC-V community has ratified several ISA extensions creating a viable path for the co-design, implementation, and production of RISC-V HPC hardware. In light of availability of a path to RISC-V HPC hardware, Tactical Computing Labs will present an introduction to the RISC-V, highlight portions of the ISA specification most relevant to the HPC community, present an overview of the current state of RISC-V HPC software,  and identify avenues for the HPC runtime system community to engage with the RISC-V community.
\end{conf-abstract}

\chapter{Session Chairs}

TODO

\begin{itemize}
\item Keynote 1, Patrick Diehl
\item Session 1 (Wed 10:30 to 12:20), Hartmut Kaiser 
\item Session 2 (Wed 2:00 to 3:30), Gerald Baumgartner
\item Keynote 2, Hartmut Kaiser
\item Session 3 (Thu 10:00 to 12:30), Patrick Diehl
\item Session 4 (Thu 2:00 to 3:30), Patricia Grubel
\item Keynote 3, Steven R. Brandt
\item Session 5 (Fr 10:30 to 12:30), Steven R. Brandt 
\end{itemize}

\chapter{Talks}
\textcolor{blue}{Note that blue names indicate virtual talks.}

\DTLforeach*{data}{\last=Last,\first=First,\affiliation=Affiliation,\title=Title,\datum=Date,\time=Time,\text=Abstract}
{
\begin{conf-abstract}[\datum\\\tiny\time]
{\title}
{\first~ \last}
{\affiliation}
\indexauthors{\last~\first}
\input{\text}
\newpage
\end{conf-abstract}
}

\chapter{Posters}

\DTLforeach*{dataPoster}{\last=Last,\first=First,\affiliation=Affiliation,\title=Title,\datum=Date,\time=Time,\text=Abstract}
{
\def\x{\last}
\def\y{\substring{\x}{1}{1}\par}
\begin{conf-abstract}[\datum\\\time]
{\title}
{\first~\last}
{\affiliation}
\indexauthors{\last~\first}
\begin{center}
\input{\text}
\end{center}
\end{conf-abstract}
}


% Specify conf-abstract like this:
% \begin{conf-abstract}[optional text going into the margin note]
% {Title of Paper}
% {Authors (use \textsuperscript as institution markers)}
% {Institutions (use \textsuperscript as institution markers)}
% \indexauthors{Lastname1!Firstname 1, Lastname2!Firstname2}
% Abstract text
% \end{conf-abstract}
%
% It's probably best to generate the abstracts from a 
% database or something via a script. Don't forget to
% check through for any special characters that need to
% be escaped.

%\input{abstracts/paper1}
%\input{abstracts/paper2}
%\input{abstracts/paper3}

\chapter{Additional information}

\section{Addresses}

TODO: Joseph

\subsection*{Workshop venue}
Digital Media Center, Center for Computation \& Technology, 340 E Parker Blvd., Baton Rouge, LA 70803
\subsection*{Hotel}
The Cook Hotel at LSU, 3848 W Lakeshore Dr, Baton Rouge, LA 70808 \\
(225) 383-2665
\subsection*{Banquet}
LSU Faculty Club, 101 Tower Dr, Baton Rouge, LA 70803 \\
(225) 578-2356

\section{Restaurants}

\subsection*{Walking distance}

\begin{itemize}
\item The Chimes -- Lively campus-area hangout from a local chain featuring a worldwide beer list \& hearty bar fare: 3357 Highland Rd, Baton Rouge, LA 70802 (225 383-1754)
\item Louie's Cafe -- LSU-area fixture dating to 1941 serves a diner menu 24/7 in a classic lunch-counter setting: 3322 Lake St, Baton Rouge, LA 70802 (225 346-8221)
\item Highland Coffees -- Charming, airy locale with a laid-back vibe for coffee roasted on-site \& a variety of baked goods: 3350 Highland Rd, Baton Rouge, LA 70802 (225 336-9773)
\end{itemize}

\subsection*{Local cuisine}

\begin{itemize}
\item Parrain's Seafood Restaurant -- Local seafood specialist cooking up Louisiana recipes in a rustic space with porch seating: 3225 Perkins Rd, Baton Rouge, LA 70808 (225 381-9922)
\item Mike Anderson's - Baton Rouge -- Area staple for regional seafood in a spacious, wood-lined setting with a sports-friendly vibe: 1031 W Lee Dr, Baton Rouge, LA 70820 (225 766-7823)
\item Stroubes Seafood and Steaks -- Chophouse presenting local preparations of meat \& seafood in comfortable digs with a lounge: 107 3rd St, Baton Rouge, LA 70801 (225 448-2830)
\end{itemize}


\backmatter
\renewcommand{\indexname}{Author Index}
\printindex
\vspace{2cm}
\doclicenseThis 

\end{document}
